\documentclass[10pt,landscape]{article}
\usepackage{multicol}
\usepackage{calc}
\usepackage{ifthen}
\usepackage[landscape]{geometry}
\usepackage{hyperref}
\usepackage{amsmath}
\usepackage{amsthm} %needed for the proofs 
\usepackage{amssymb}


\ifthenelse{\lengthtest { \paperwidth = 11in}}
	{ \geometry{top=.5in,left=.5in,right=.5in,bottom=.5in} }
	{\ifthenelse{ \lengthtest{ \paperwidth = 297mm}}
		{\geometry{top=1cm,left=1cm,right=1cm,bottom=1cm} }
		{\geometry{top=1cm,left=1cm,right=1cm,bottom=1cm} }
	}

% Turn off header and footer
\pagestyle{empty}
 

% Redefine section commands to use less space
\makeatletter
\renewcommand{\section}{\@startsection{section}{1}{0mm}%
                                {-1ex plus -.5ex minus -.2ex}%
                                {0.5ex plus .2ex}%x
                                {\normalfont\large\bfseries}}
\renewcommand{\subsection}{\@startsection{subsection}{2}{0mm}%
                                {-1explus -.5ex minus -.2ex}%
                                {0.5ex plus .2ex}%
                                {\normalfont\normalsize\bfseries}}
\renewcommand{\subsubsection}{\@startsection{subsubsection}{3}{0mm}%
                                {-1ex plus -.5ex minus -.2ex}%
                                {1ex plus .2ex}%
                                {\normalfont\small\bfseries}}
\makeatother

% Define BibTeX command
\def\BibTeX{{\rm B\kern-.05em{\sc i\kern-.025em b}\kern-.08em
    T\kern-.1667em\lower.7ex\hbox{E}\kern-.125emX}}

% Don't print section numbers
\setcounter{secnumdepth}{0}


\setlength{\parindent}{0pt}
\setlength{\parskip}{0pt plus 0.5ex}


% -----------------------------------------------------------------------

\begin{document}

\raggedright
\footnotesize
\begin{multicols}{3}


% multicol parameters
% These lengths are set only within the two main columns
%\setlength{\columnseprule}{0.25pt}
\setlength{\premulticols}{1pt}
\setlength{\postmulticols}{1pt}
\setlength{\multicolsep}{1pt}
\setlength{\columnsep}{2pt}

\begin{center}
     \Large{\textbf{MATH 255 Cheat Sheet}} \\
\end{center}

\section{Lecture Notes 1}
\subsection{Definitions}
\begin{enumerate}
	\item Cluster/limit point : Every $\varepsilon$-neighbourhood of $x$ contains a point of $S$, i.e. every neighbourhood contains infinitely many points, i.e. there exists a sequence in $S$ which converges to $x$.
	\item Closed set $\iff$ contains all its cluster points
	\item Interior point, i.e. $x\in S^o$ if $\exists$ $\varepsilon$ such that $B(x, \varepsilon) \subseteq S$
	\item Isolated point if $\exists \varepsilon$ s.t. $B(x, \varepsilon) \cap S = \{x\}$
	\item Boundary point if $\forall \varepsilon, $ $B(x, \varepsilon)\cap S \neq \emptyset$ and $B(x, \varepsilon)\cap S^c \neq \emptyset$
	\item Closure of a set $\overline{S} = S \cup \partial S = S \cup S'$ 
	\item \textbf{Compact} if $\{G_\alpha\}_{\alpha \in I}$ is an open cover of $S$, $\exists$ a finite subcover s.t. $S\subseteq G_{\alpha_1} \cup ... \cup G_{\alpha_n}$
	\item Continuity:
\end{enumerate}
\subsection{Results}
\begin{enumerate}
	\item $K_n$ a sequence of compact sets s.t. $K_{n-1} \subseteq K_n$, then the intersection of all $K_n$ is compact and non-empty.
	\item Perfect $\implies$ uncountable.
	
\end{enumerate}

\section{Lecture Notes 2 - Metric Spaces}
\subsection{Definitions}
\begin{enumerate}
	\item \textbf{Metric space} $X$:
	\begin{enumerate}
		\item $d(x,y) \geq 0 \forall x,y \in X$
		\item $d(x,y) = 0 \iff x=y$
		\item $d(x,y) = d(y,x)$
		\item $d(x,y) \leq d(x,z) + d(z,y) \forall x,y,z \in X$
	\end{enumerate}
	\item Open ball in $X$: $B(x,\varepsilon) := \{ y \in X \; : \; d(x,y) < \varepsilon \} $
	\item $S$ open in $X$ if $\forall x \in S, \; \exists \varepsilon >0 $ s.t. $\{ y\in X \mid d(x,y) <\varepsilon \}\subseteq S$
	\item Perfect in $X$ if closed and every point is a cp.
	\item $E \subseteq X$ is bounded if $\exists x \in X$ and $R> 0$ s.t. $\forall y \in E, \; d(x,y) < R$. 
	\item S is dense in $X$ if $\overline{S} = X$, i.e. every $x\in S$ is a cp of $X$, i.e. $\forall x \in X, \forall \varepsilon>0, \; \exists$ a point of S in $B(x, \varepsilon)$. 
	\item $X$ is separable if it has a countable dense subset.
	\item $x \in X$ is a condensation point if $\forall \varepsilon>0, \; \exists$ uncountably many points of X in $B(x, \varepsilon)$.
	\item $K\subseteq X$ is \textbf{sequentially compact} if every infinite subset E of K has a cluster point in K. That is, every sequence in K has a subsequence converging in K.
	\item A set $S\subseteq X$ is \textbf{totally bounded} if $\forall \varepsilon > 0, \; \exists $ finitely many $x_n \in S$ s.t. $S\subseteq B(x_1, \varepsilon) \cup ... \cup B(x_N, \varepsilon)$. 
	\item A collections of subsets of $E$ labeled as $\mathcal{F}$ has the \textbf{FIP} if whenever $F_1,...,F_n \in \mathcal{F}$, we have
	$$\cap^n_{i=1}F_i \neq \emptyset$$

\end{enumerate}
\subsection{Results} 
\begin{enumerate}
	\item The union of arbitrary open sets is open.
	\item The union of finitely many closed sets is closed.
	\item The intersection of arbitrary closed sets is closed.
	\item The intersection of finitely many open sets is open.
	\item $E \subseteq Y \subseteq X$. Then $E$ is open relative to $Y \iff \exists G$ open in $X$ s.t. $E = G\cap Y$.
	\item $f:E \rightarrow \mathbb{R}$ is continuous on $E$ if the inverse image of any open set in $\mathbb{R}$ is open relative to $E$. 
	\item $K \subseteq Y \subseteq X$ Then K is compact relative to X $\iff$ it is compact  relative to Y.
	\item Compact $\implies$ closed \& bounded (in any metric space).
	\item Closed subsets of compact sets are compact.
	\item F closed, K compact $\implies F\cap K$ compact.
	\item \textbf{Sequentially Compact} $\iff$ \textbf{Compact}.
	\item $K \subseteq X$, K is compact $\iff$ K is closed and every collection $\mathcal{F}$ of closed subsets of K which has the FIP satisfies $\cap_{F\in\mathcal{F}}F_i \neq \emptyset$
	\item Totally bounded $\implies$ separable.
	\item Sequentially compact $\implies$ separable.

	
\end{enumerate}

\section{Lecture Notes 3 - Sequences \& Continuous Functions in Metric Spaces}

\section{Lecture Notes 4 - Normed Vector Spaces}

\section{Lecture Notes 5 - Infinite Series}

\section{Lecture Notes 6 - Integration}

\end{multicols}
\end{document}
