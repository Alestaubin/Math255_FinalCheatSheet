\documentclass[10pt,landscape]{article}
\usepackage{multicol}
\usepackage{calc}
\usepackage{ifthen}
\usepackage[landscape]{geometry}
\usepackage{hyperref}
\usepackage{amsmath}
\usepackage{amsthm} %needed for the proofs 
\usepackage{amssymb}


\ifthenelse{\lengthtest { \paperwidth = 11in}}
	{ \geometry{top=.5in,left=.5in,right=.5in,bottom=.5in} }
	{\ifthenelse{ \lengthtest{ \paperwidth = 297mm}}
		{\geometry{top=1cm,left=1cm,right=1cm,bottom=1cm} }
		{\geometry{top=1cm,left=1cm,right=1cm,bottom=1cm} }
	}

% Turn off header and footer
\pagestyle{empty}
 

% Redefine section commands to use less space
\makeatletter
\renewcommand{\section}{\@startsection{section}{1}{0mm}%
                                {-1ex plus -.5ex minus -.2ex}%
                                {0.5ex plus .2ex}%x
                                {\normalfont\large\bfseries}}
\renewcommand{\subsection}{\@startsection{subsection}{2}{0mm}%
                                {-1explus -.5ex minus -.2ex}%
                                {0.5ex plus .2ex}%
                                {\normalfont\normalsize\bfseries}}
\renewcommand{\subsubsection}{\@startsection{subsubsection}{3}{0mm}%
                                {-1ex plus -.5ex minus -.2ex}%
                                {1ex plus .2ex}%
                                {\normalfont\small\bfseries}}
\makeatother

% Define BibTeX command
\def\BibTeX{{\rm B\kern-.05em{\sc i\kern-.025em b}\kern-.08em
    T\kern-.1667em\lower.7ex\hbox{E}\kern-.125emX}}

% Don't print section numbers
\setcounter{secnumdepth}{0}


\setlength{\parindent}{0pt}
\setlength{\parskip}{0pt plus 0.5ex}


% -----------------------------------------------------------------------

\begin{document}

\raggedright
\footnotesize
\begin{multicols}{3}


% multicol parameters
% These lengths are set only within the two main columns
%\setlength{\columnseprule}{0.25pt}
\setlength{\premulticols}{1pt}
\setlength{\postmulticols}{1pt}
\setlength{\multicolsep}{1pt}
\setlength{\columnsep}{2pt}

\begin{center}
     \Large{\textbf{MATH 255 Cheat Sheet}} \\
\end{center}

\section{Lecture Notes 1}
\subsection{Definitions}
\begin{enumerate}
	\item Cluster/limit point : Every $\varepsilon$-neighbourhood of $x$ contains a point of $S$, i.e. every neighbourhood contains infinitely many points, i.e. there exists a sequence in $S$ which converges to $x$.
	\item Closed set $\iff$ contains all its cluster points
	\item Interior point, i.e. $x\in S^o$ if $\exists$ $\varepsilon$ such that $B(x, \varepsilon) \subseteq S$
	\item Isolated point if $\exists \varepsilon$ s.t. $B(x, \varepsilon) \cap S = \{x\}$
	\item Boundary point if $\forall \varepsilon, $ $B(x, \varepsilon)\cap S \neq \emptyset$ and $B(x, \varepsilon)\cap S^c \neq \emptyset$
	\item Closure of a set $\overline{S} = S \cup \partial S = S \cup S'$ 
	\item \textbf{Compact} if $\{G_\alpha\}_{\alpha \in I}$ is an open cover of $S$, $\exists$ a finite subcover s.t. $S\subseteq G_{\alpha_1} \cup ... \cup G_{\alpha_n}$
	\item Continuity:
\end{enumerate}
\subsection{Results}
\begin{enumerate}
	\item $K_n$ a sequence of compact sets s.t. $K_{n-1} \subseteq K_n$, then the intersection of all $K_n$ is compact and non-empty.
	\item Perfect $\implies$ uncountable.
	
\end{enumerate}

\section{Lecture Notes 2 - Metric Spaces}
\subsection{Definitions}
\begin{enumerate}
	\item \textbf{Metric space} $X$:
	\begin{enumerate}
		\item $d(x,y) \geq 0 \forall x,y \in X$
		\item $d(x,y) = 0 \iff x=y$
		\item $d(x,y) = d(y,x)$
		\item $d(x,y) \leq d(x,z) + d(z,y) \forall x,y,z \in X$
	\end{enumerate}
	\item Open ball in $X$: $B(x,\varepsilon) := \{ y \in X \; : \; d(x,y) < \varepsilon \} $
	\item $S$ open in $X$ if $\forall x \in S, \; \exists \varepsilon >0 $ s.t. $\{ y\in X \mid d(x,y) <\varepsilon \}\subseteq S$
	\item Perfect in $X$ if closed and every point is a cp.
	\item $E \subseteq X$ is bounded if $\exists x \in X$ and $R> 0$ s.t. $\forall y \in E, \; d(x,y) < R$. 
	\item S is dense in $X$ if $\overline{S} = X$, i.e. every $x\in S$ is a cp of $X$, i.e. $\forall x \in X, \forall \varepsilon>0, \; \exists$ a point of S in $B(x, \varepsilon)$. 
	\item $X$ is separable if it has a countable dense subset.
	\item $x \in X$ is a condensation point if $\forall \varepsilon>0, \; \exists$ uncountably many points of X in $B(x, \varepsilon)$.
	\item $K\subseteq X$ is \textbf{sequentially compact} if every infinite subset E of K has a cluster point in K. That is, every sequence in K has a subsequence converging in K.
	\item A set $S\subseteq X$ is \textbf{totally bounded} if $\forall \varepsilon > 0, \; \exists $ finitely many $x_n \in S$ s.t. $S\subseteq B(x_1, \varepsilon) \cup ... \cup B(x_N, \varepsilon)$. 
	\item A collections of subsets of $E$ labeled as $\mathcal{F}$ has the \textbf{FIP} if whenever $F_1,...,F_n \in \mathcal{F}$, we have
	$$\cap^n_{i=1}F_i \neq \emptyset$$

\end{enumerate}
\subsection{Results} 
\begin{enumerate}
	\item The union of arbitrary open sets is open.
	\item The union of finitely many closed sets is closed.
	\item The intersection of arbitrary closed sets is closed.
	\item The intersection of finitely many open sets is open.
	\item $E \subseteq Y \subseteq X$. Then $E$ is open relative to $Y \iff \exists G$ open in $X$ s.t. $E = G\cap Y$.
	\item $f:E \rightarrow \mathbb{R}$ is continuous on $E$ if the inverse image of any open set in $\mathbb{R}$ is open relative to $E$. 
	\item $K \subseteq Y \subseteq X$ Then K is compact relative to X $\iff$ it is compact  relative to Y.
	\item Compact $\implies$ closed \& bounded (in any metric space).
	\item Closed subsets of compact sets are compact.
	\item F closed, K compact $\implies F\cap K$ compact.
	\item \textbf{Sequentially Compact} $\iff$ \textbf{Compact}.
	\item $K \subseteq X$, K is compact $\iff$ K is closed and every collection $\mathcal{F}$ of closed subsets of K which has the FIP satisfies $\cap_{F\in\mathcal{F}}F_i \neq \emptyset$
	\item Totally bounded $\implies$ separable.  
	\item Sequentially compact $\implies$ separable.
\end{enumerate}

\section{Lecture Notes 3 - Sequences \& Continuous Functions in Metric Spaces}

\begin{enumerate}
	\item $a_n$ converges to $a$ if $\forall \varepsilon >0, \; \exists N$ s.t. $n\geq N \implies d(a_n, a) < \varepsilon$
	
	\item X a compact metric space $\implies$ every sequence in X has a convergent subsequence to a point in X.
	
	\item $p_n \in X$ cpt m.s. then the set of subsequential limits is closed.
	
	\item \textbf{Cauchy sequence}: A sequence $p_n$ in a metric space $(X, d)$ is Cauchy if for every $\varepsilon >0, \; \exists N$ s.t. $n\geq m\geq N \implies d(p_n, p_m) < \varepsilon$
	
	\item Convergent $\implies$ Cauchy.
	\item $X$ cpt $\implies$ every Cauchy sequence converges to a point in $X$.
	
	\item \textbf{Complete} m.s. if every Cauchy sequence converges.
	
	\item $E\subseteq X, \; f: E\rightarrow Y$, $\lim_{n\rightarrow p} f(x) = q$ if $\forall \varepsilon>0, \exists \delta$ s.t. 
	$$d_X(x, p) < \delta \implies d_Y(f(x), q ) < \varepsilon$$
	
	\item $\lim_{x\rightarrow p} f(x) = q \iff \forall p_n \neq p$ s.t. $\lim_{n\rightarrow \infty} p_n = p$, we have $\lim_{n\rightarrow \infty} f(p_n) = q$.
	
	\item $f: X\rightarrow Y$ cts at p $\forall \varepsilon> 0, \exists \delta >0$ s.t. if $ d_X(x, p) < \delta \implies d_Y(f(x), f(p)) < \varepsilon$.
	
	\item $f: X\rightarrow Y$ is cts $\iff \forall$ open set $V \in Y$, $f^{-1}(V)$ is open in $X$.
	
	\item $f: X \rightarrow Y$ is cts $\iff \forall p_n \rightarrow p \in X, $ we have $f(p_n) \rightarrow f(p) \in Y$. 
	
	\item Let $X$ cpt m.s., if $f: X \rightarrow Y$ is cts, then $f(X) \subseteq Y$ is cpt.
\end{enumerate}

\section{Lecture Notes 4 - Normed Vector Spaces}
\begin{enumerate}
	\item \textbf{Minkowsky inequality}: Let $x = (x_1, ..., x_N)$, $ y = (y_1, ..., y_N) \in \mathbb{R}^n$ and let $1 \leq p < \infty$ then, 

$
\left(\sum^N_{n=1} ( |x_n|+|y_n| )^p \right)^{1/p}  \leq \left( \sum^N_{n=1} |x_n|^p \right)^{1/p}+ \left( \sum^N_{n=1} |y_n|^p \right)^{1/p} 
= \|x\|_p +\|y\|_p
$
	\item \textbf{Holder Inequality}: let $x,y$ be as above, $1<p,q<\infty$ be conjugate exponents $(p^{-1}+ q^{-1} = 1)$ then, 
	
$\sum^N_{n=1} |x_n| |y_n| \leq \left( \sum^N_{n=1} |x_n|^p \right)^{1/p}\left( \sum^N_{n=1} |y_n|^q \right)^{1/q} = \| x\|_p \| y\|_q $
	\item $\|\cdot \|_A $ and $\| \cdot\|_B$ are \textbf{equivalent norms} if $\exists c_1, c_2>0$ s.t. $\forall x \in X,$ $\| x\|_A\leq c_1 \| x\|_B$ and $\|x\|_B \leq c_2\|x\|_A$
	\item All norms are equivalent in $\mathbb{R}^N$ and on finite v.s. 
	\item $T: X \rightarrow Y$ is bounded if $\exists c>0$ s.t. $\|T(x)\|_Y \leq c\|x\|_X \; \forall x \in X$.
	\item $T: X\rightarrow Y$, TFAE:
	
	a) T is bounded
	
	b) T is cts at all points in X
	
	c) T is cts at $x=0$
	\item X f.d. v.s. $\implies T:X \rightarrow Y $ is bounded.
	\item \textbf{Operator norm}: $\| T\|_{op} = \sup_{x\neq 0} \frac{ \|T(x) \|_Y}{\| x \|_X} = \sup_{\|x\|_X = 1} \|T(x)\|_Y = \inf \{ c\geq 0 \mid \|T(x)\|_Y\leq c\|x\|_X \forall x\} $
	\item $F: X\rightarrow Y$, $U\subseteq X$ open, $p \in U$, then $F$ is differentiable at $p$ if $\exists$ bounded linear operator $T: X \rightarrow Y$ s.t.
	$$\lim_{h\rightarrow 0} \frac{\|F(p+h) -F(p) -T(h) \|_Y}{\|h\|_X} = 0 $$
	Then, $T$ is the \textbf{derivative} of $F$ at $p$. 
	 
\end{enumerate}


\section{Lecture Notes 5 - Infinite Series}
\begin{enumerate}
	\item $\sum^\infty_{k=1} a_k $ converges $\iff$ the sequence of partial sums $s_n$ converges, where $s_n := a_1 + ... + a_n$.
	\item Geometric series $\sum^\infty_{k=0} r^n $ converges to $\frac{1}{1-r}$ if $r<1$, diverges o.w.
	\item Harmonic series $\sum^\infty_{k=1}\frac{1}{k}$ diverges.
	\item $p$-series $\sum^\infty_{k=1}\frac{1}{k^p}$ converges $ \iff 1<p<\infty$.
	\item $\sum a_k$ converges $ \implies \lim_{k\rightarrow \infty} a_k = 0$.
	\item Cauchy criterion for convergence: $\sum a_k$ converges $\iff \forall \varepsilon >0, \; \exists M\in \mathbb{N} $ s.t. $m>n\geq M  \implies |s_m -s_n| = |a_{n+1} + ...+ a_m | < \varepsilon$
	\item Direct comparison test: $0\leq a_k\leq b_k \; \forall k>K$, then
	
	a) $\sum b_k$ converges $\implies \sum a_k$ converges.
	
	b) $\sum a_k$ diverges $\implies \sum b_k$ diverges.
	\item Limit comparison test 1: $0<a_k, b_k $ s.t. $r = \lim_{k\rightarrow \infty} a_k/b_k$
	
	a) $r\neq 0$: $\sum a_k$ converges $\iff \sum b_k$ converges.
	
	b) $r = 0$: $\sum b_k$ converges $\implies \sum a_k$ converges. 
	\item Limit comparison test 2: $a_k, b_k $ s.t. $r = \lim_{k\rightarrow \infty} |a_k|/|b_k|$
	
	a) $r\neq 0$: $\sum a_k$ converges abs. $\iff \sum b_k$ conv. abs.
	
	b) $r = 0$: $\sum b_k$ converges abs. $\implies \sum a_k$ converges abs. 

	
	\item $\sum a_k$ converges $\implies$ every regrouping converges.
	
	\item Absolute convergence $\implies $ convergence
	
	\item Ratio test: let $a_k \neq 0$, s.t. $\lim_{k \rightarrow \infty} \frac{|a_{k+1}|}{ |a_k|} = r $, then if
	
	a) $r<1, \; \sum a_k$ converges abs.
	
	b) $r>0, \; \sum a_k$ diverges.
	
	c) $r=0$, inconclusive.  
	
	\item Alternating series test: $a_k\geq 0 $ non-increasing converging to 0. Then $\sum (-1)^{k+1} a_k$ converges. 

	
	\item \textbf{Dirichlet's test}: $a_k$ a decreasing sequence s.t. $\lim_{n\rightarrow \infty} a_k = 0$ and $b_k$ s.t. the partial sums of $b_k$ are bounded. Then $\sum a_k b_k$ converges.
	
	\item \textbf{Abel's test}: $a_k$ a convergent monotone sequence, $\sum b_k$ converges, then $\sum a_k b_k $ converges. 
	
	\item $a_k$ converges absolutely $\implies$ every rearrangement of $a_k$ converges to the same point. 
	\item $a_k$ converges conditionally, $\alpha$ any real number. Then there exists a rearrangement of $a_k$ which converges to $\alpha$.
	 
\end{enumerate}

\section{Lecture Notes 6 - Integration}

\begin{enumerate}
	\item Tagged partition $\dot{P}$ on $[a,b]$ is defined as $\{[x_{i-1}, x_i ], t_i\}^n_{i=0}$ where $t_i$ is the point chosen for the subinterval.
	\item $\|\dot{P} \| = \max \{[x_{i-1}, x_i ]\}^n_{i=0}$
	\item \textbf{Riemann sum} $S(f, \dot{P}) := \sum^n_{i=1} f(t_i)( x_i -x_{i-1})$.
	
	$$\lim_{n\rightarrow \infty}\sum^n_{i=1} f(x_i) \Delta x = \int^b_a f(x)dx$$
	where $\Delta x = \frac{b-a}{n}$ and $x_i = a + i\Delta x$
	\item $f \in R[a,b]$ if $\exists L \in \mathbb{R}$ s.t. $\forall \varepsilon > 0, \exists \delta_\varepsilon$ s.t. if $\|\dot{P}\| < \delta_\varepsilon$, $$|S(f, \dot{P})- L| < \varepsilon$$
	Then, $L$ is the \textit{unique} integral. 
	\item Let $f,g \in R[a,b]$, then
	
	a) $\int^b_a kf = k \int^b_a f$
	
	b) $\int^b_a f+g = \int^b_a f + \int^b_a g $
	
	c) $f(x) \leq g(x) \forall x \in [a,b] \implies \int^b_a f \leq \int^b_a g$
	6
	\item $f\in R[a,b] \implies f $ bounded on $[a,b]$ 
	
	\item \textbf{Cauchy criterion}: $f \in R[a,b] \iff \forall \varepsilon >0, \exists \mu_\varepsilon$ s.t. if $\|\dot{P} \| , \| \dot{Q}\| < \mu_\varepsilon$, then 
	$$| S(f, \dot{P} ) - S(f, \dot{Q} ) | \leq \varepsilon$$
	
	\item \textbf{Squeeze}: $f\in R[a,b] \iff \forall \varepsilon > 0, \exists \alpha, \omega \in R[a,b]$ s.t. 
	$$\alpha(x) \leq f(x) \leq \omega(x) \quad \forall x \in[a,b]$$
	and such that
	$$\int^b_a (\omega- \alpha ) < \varepsilon $$
	
	\item $f:[a,b] \rightarrow \mathbb{R}$ cts $\implies$ $f\in R[a,b]$.
	\item $f\in R[a,b] \implies \forall c \in [a,b], \; f|_{[a,c]} \land f|_{[c,b]}$ are Riemann integrable and in particular
	$$\int^b_a f = \int^c_a f + \int^b_c f$$
\end{enumerate}

\end{multicols}
\end{document}







